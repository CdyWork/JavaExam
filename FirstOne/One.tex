\documentclass{article}
\usepackage[UTF8]{ctex}

\begin{document}
Hey world!

测试

This is a first document.
\end{document}

\geometry{a4paper, margin=1in}

\title{Java环境配置与控制台程序设计实验报告}
\author{\underline{__________}(学号:\underline{__________})}
\date{\underline{__________年__________月__________日}}

\begin{document}
\maketitle

\section{实验基本信息}
\begin{center}
\begin{tabular}{|m{3cm}<{\centering}|m{10cm}<{\centering}|}
\hline
实验名称 & Java环境配置与控制台程序设计 \\
\hline
实验地点 & \underline{__________} \\
\hline
实验时长 & \underline{__________小时} \\
\hline
参考资料 & 《Java环境安装和配置指南》、网盘提供的jdk1.8.0_212.zip与eclipse.zip \\
\hline
\end{tabular}
\end{center}

\section{实验目的}
\begin{enumerate}
    \item 掌握SDK软件包、Eclipse软件、EditPlus编辑软件的安装方法。
    \item 掌握程序运行环境的设置方法。
    \item 掌握Java程序的编写与运行方法。
    \item 理解面向对象的编程思想。
\end{enumerate}

\section{实验准备}
\begin{enumerate}
    \item 软件资源:从指定网盘下载并获取\textbf{jdk1.8.0_212.zip}压缩包与\textbf{eclipse.zip}压缩包。
    \item 工具准备:计算机需安装解压软件(如WinRAR、7-Zip),用于提取上述压缩包中的文件。
    \item 文档准备:提前阅读《Java环境安装和配置指南》,明确软件安装与环境配置的关键步骤。
\end{enumerate}

\section{实验内容与步骤}
\subsection{一、软件安装与环境配置}
\subsubsection{1. SDK(JDK)安装与环境变量配置}
\begin{enumerate}
    \item 解压JDK压缩包:右键点击\textbf{jdk1.8.0_212.zip},选择解压路径(建议路径:\underline{__________},如C:\textbackslash Java\textbackslash jdk1.8.0_212),完成解压。
    \item 配置系统环境变量:
          \begin{enumerate}
              \item 右键点击“此电脑”→“属性”→“高级系统设置”→“环境变量”,进入环境变量配置界面。
              \item 新建系统变量\textbf{JAVA_HOME}:变量名填写“JAVA_HOME”,变量值填写JDK解压路径(如C:\textbackslash Java\textbackslash jdk1.8.0_212)。
              \item 编辑系统变量\textbf{Path}:在“Path”变量值中新增“\%JAVA_HOME\%\textbackslash bin”与“\%JAVA_HOME\%\textbackslash jre\textbackslash bin”两个路径。
              \item 新建系统变量\textbf{CLASSPATH}:变量名填写“CLASSPATH”,变量值填写“.;\%JAVA_HOME\%\textbackslash lib\textbackslash dt.jar;\%JAVA_HOME\%\textbackslash lib\textbackslash tools.jar”(注意开头的“.”不可省略)。
          \end{enumerate}
    \item 验证JDK配置:打开命令提示符(CMD),分别输入“java -version”与“javac -version”,若显示JDK版本信息(如java version "1.8.0_212"),则配置成功。
          \begin{center}
              \includegraphics[width=0.8\textwidth]{jdk_verify_screenshot} % 此处插入JDK验证截图(需含时间戳)
              \caption{JDK配置验证结果截图(时间戳:\underline{__________})}
          \end{center}
\end{enumerate}

\subsubsection{2. Eclipse软件安装与启动}
\begin{enumerate}
    \item 解压Eclipse压缩包:右键点击\textbf{eclipse.zip},选择解压路径(建议路径:\underline{__________},如D:\textbackslash Eclipse),完成解压。
    \item 启动Eclipse:双击解压路径下的“eclipse.exe”,选择工作空间(可默认或自定义路径:\underline{__________}),等待软件启动完成。
          \begin{center}
              \includegraphics[width=0.8\textwidth]{eclipse_start_screenshot} % 此处插入Eclipse启动截图(需含时间戳)
              \caption{Eclipse启动界面截图(时间戳:\underline{__________})}
          \end{center}
\end{enumerate}

\subsubsection{3. EditPlus编辑软件安装}
\begin{enumerate}
    \item 运行EditPlus安装程序:双击下载的EditPlus安装包,按照向导提示选择安装路径(如C:\textbackslash Program Files\textbackslash EditPlus)、创建桌面快捷方式。
    \item 启动与基础设置:双击桌面“EditPlus”快捷方式,根据需求设置编码格式(建议UTF-8)、字体大小等,完成基础配置。
          \begin{center}
              \includegraphics[width=0.8\textwidth]{editplus_setup_screenshot} % 此处插入EditPlus配置截图(需含时间戳)
              \caption{EditPlus基础配置截图(时间戳:\underline{__________})}
          \end{center}
\end{enumerate}

\subsection{二、简单控制台应用程序开发}
\subsubsection{1. 使用Eclipse实现(输出“Hello World!”与“We are students.”)}
\begin{enumerate}
    \item 创建Java项目:打开Eclipse,点击“File”→“New”→“Java Project”,输入项目名(如myFirstJava),点击“Finish”。
    \item 创建Java类:右键点击项目“myFirstJava”→“New”→“Class”,输入类名(HelloWorld),勾选“public static void main(String[] args)”,点击“Finish”。
    \item 编写代码:在HelloWorld.java文件中编写如下代码(需包含包声明):
          \begin{verbatim}
package myFirstJava; // 包声明,与项目结构对应

public class HelloWorld {
    public static void main(String[] args) {
        System.out.println("Hello World!");
        System.out.println("We are students.");
    }
}
          \end{verbatim}
    \item 编译与运行:点击菜单栏“Run”→“Run”(或快捷键Ctrl+F11),查看控制台输出结果,截图保存。
          \begin{center}
              \includegraphics[width=0.8\textwidth]{eclipse_run_screenshot} % 此处插入Eclipse运行结果截图(需含时间戳)
              \caption{Eclipse中程序运行结果截图(时间戳:\underline{__________})}
          \end{center}
\end{enumerate}

\subsubsection{2. 使用命令行方式实现}
\begin{enumerate}
    \item 编写Java文件:打开EditPlus,编写与上述步骤3相同的代码,将文件保存至指定目录(需与包声明对应,如C:\textbackslash Java\textbackslash Test\textbackslash myFirstJava,文件名为HelloWorld.java)。
    \item 编译Java文件:打开CMD,通过“cd”命令切换至包目录的上级路径(如cd C:\textbackslash Java\textbackslash Test),输入“javac myFirstJava/HelloWorld.java”,若编译成功,在myFirstJava目录下生成HelloWorld.class文件,截图保存。
          \begin{center}
              \includegraphics[width=0.8\textwidth]{cmd_compile_screenshot} % 此处插入命令行编译截图(需含时间戳)
              \caption{命令行编译Java文件截图(时间戳:\underline{__________})}
          \end{center}
    \item 运行class文件:在CMD中输入“java myFirstJava.HelloWorld”,查看输出结果,截图保存。
          \begin{center}
              \includegraphics[width=0.8\textwidth]{cmd_run_screenshot} % 此处插入命令行运行截图(需含时间戳)
              \caption{命令行运行程序结果截图(时间戳:\underline{__________})}
          \end{center}
\end{enumerate}

\subsection{三、Eclipse中错误提示实验(思考与记录)}
\subsubsection{1. 丢失大括号的错误提示}
\begin{enumerate}
    \item 操作:在HelloWorld.java中删除main方法体后的“}”(如删除“System.out.println("We are students.");”后的“}”),保存并编译。
    \item 错误提示记录:\underline{__________}(如“Syntax error, insert "}" to complete Block”)
    \item 截图保存:
          \begin{center}
              \includegraphics[width=0.8\textwidth]{brace_error_screenshot} % 此处插入丢失大括号错误截图(需含时间戳)
              \caption{丢失大括号的错误提示截图(时间戳:\underline{__________})}
          \end{center}
\end{enumerate}

\subsubsection{2. 语句丢失分号的错误提示}
\begin{enumerate}
    \item 操作:在HelloWorld.java中删除某条System.out.println语句后的分号(如删除“System.out.println("Hello World!")”后的“;”),保存并编译。
    \item 错误提示记录:\underline{__________}(如“Syntax error, insert ";" to complete Statement”)
    \item 截图保存:
          \begin{center}
              \includegraphics[width=0.8\textwidth]{semicolon_error_screenshot} % 此处插入丢失分号错误截图(需含时间戳)
              \caption{丢失分号的错误提示截图(时间戳:\underline{__________})}
          \end{center}
\end{enumerate}

\subsubsection{3. 将System写成system的错误提示}
\begin{enumerate}
    \item 操作:在HelloWorld.java中把“System.out.println”改为“system.out.println”,保存并编译。
    \item 错误提示记录:\underline{__________}(如“system cannot be resolved to a variable”)
    \item 截图保存:
          \begin{center}
              \includegraphics[width=0.8\textwidth]{system_case_error_screenshot} % 此处插入System大小写错误截图(需含时间戳)
              \caption{System大小写错误的提示截图(时间戳:\underline{__________})}
          \end{center}
\end{enumerate}

\subsection{四、面向对象与面向过程概念理解}
\subsubsection{1. 核心概念定义}
\begin{itemize}
    \item 面向过程:\underline{__________}(如“以解决问题的步骤为核心,将问题分解为多个步骤,通过函数实现每个步骤,按顺序执行函数完成任务”)
    \item 面向对象:\underline{__________}(如“以构成问题的事务为核心,将事务抽象为对象,通过对象的属性与方法描述事务,通过对象间的交互完成任务”)
\end{itemize}

\subsubsection{2. 五子棋游戏设计思路对比}
\begin{center}
\begin{tabular}{|m{5cm}<{\centering}|m{7cm}<{\centering}|}
\hline
设计思想 & 具体思路 \\
\hline
面向过程 & 1. 开始游戏;2. 黑子先走;3. 绘制画面;4. 判断输赢;5. 轮到白子;6. 绘制画面;7. 判断输赢;8. 返回步骤2;9. 输出结果;将每个步骤用独立函数实现,按顺序调用函数。 \\
\hline
面向对象 & 1. 抽象对象:玩家对象(黑白双方,负责接收输入)、棋盘对象(负责绘制画面)、规则对象(负责判定输赢与犯规);2. 交互逻辑:玩家对象告知棋盘对象棋子变化→棋盘对象更新画面→规则对象判定棋局状态,循环直至游戏结束。 \\
\hline
\end{tabular}
\end{center}

\subsubsection{3. 其他案例对比(举例说明)}
\begin{enumerate}
    \item 案例描述:\underline{__________}(如“学生成绩管理系统:实现成绩录入、成绩查询、成绩统计功能”)
    \item 面向过程思路:\underline{__________}(如“1. 编写成绩录入函数;2. 编写成绩查询函数;3. 编写成绩统计函数;4. 按‘录入→查询→统计’的顺序调用函数,完成管理任务”)
    \item 面向对象思路:\underline{__________}(如“1. 抽象对象:学生对象(属性:学号、姓名、成绩;方法:成绩设置与获取)、成绩管理对象(方法:录入、查询、统计);2. 交互逻辑:成绩管理对象调用学生对象的方法获取/设置成绩,实现管理功能”)
\end{enumerate}

\subsubsection{4. 面向对象基本概念、设计特点与优点}
\begin{itemize}
    \item 基本概念:\underline{__________}(如“对象、类、属性、方法、封装、继承、多态”)
    \item 设计特点:\underline{__________}(如“以对象为核心、强调对象交互、封装性、继承性、多态性”)
    \item 设计优点:\underline{__________}(如“代码复用性高、扩展性强、可读性好、便于维护、适合大型项目开发”)
\end{itemize}

\subsubsection{5. 选做内容:DCloud与APICloud了解}
\begin{center}
\begin{tabular}{|m{5cm}<{\centering}|m{7cm}<{\centering}|}
\hline
平台名称 & 核心信息 \\
\hline
DCloud(含uni-app) & 官网:https://www.dcloud.net.cn/、https://uniapp.dcloud.net.cn/;特点:\underline{__________}(如“跨端开发框架,支持一次编写多端运行,适配小程序、APP、H5,开发效率高”);若开发小APP:开发目标\underline{__________},开发步骤\underline{__________},成果截图\underline{__________}(需含时间戳)。 \\
\hline
APICloud(用友YonBuilder) & 官网:https://developer.yonyou.com/home;特点:\underline{__________}(如“被用友收购,集成至YonBuilder低代码平台,提供数据设计、页面设计、API设计等可视化开发能力,降低开发门槛”);若开发小APP:开发目标\underline{__________},开发步骤\underline{__________},成果截图\underline{__________}(需含时间戳)。 \\
\hline
\end{tabular}
\end{center}

\section{实验问题与解决方法}
\begin{enumerate}
    \item 问题1:命令行运行时提示“找不到主类 myFirstJava.HelloWorld”
          \begin{itemize}
              \item 问题原因:\underline{__________}(如“运行路径错误,未在包目录的上级路径执行java命令;或包声明与实际目录结构不匹配”)
              \item 解决方法:\underline{__________}(如“通过cd命令切换至包目录的上级路径;检查包声明与文件保存路径是否一致,确保‘包名.类名’的调用格式正确”)
          \end{itemize}
    \item 问题2:\underline{__________}(如“Eclipse中运行程序提示‘Class not found’”)
          \begin{itemize}
              \item 问题原因:\underline{__________}(如“项目未正确编译,或类文件被删除”)
              \item 解决方法:\underline{__________}(如“右键点击项目→‘Clean’,重新编译项目;检查src目录下是否存在对应的.class文件,若不存在则重新编写代码并保存编译”)
          \end{itemize}
\end{enumerate}

\section{实验心得}
\underline{__________}(如“1. 通过本次实验,我掌握了JDK、Eclipse、EditPlus的安装与配置方法,理解了Java环境变量的作用——JAVA_HOME指定JDK路径,Path确保命令行能找到java与javac命令,CLASSPATH指定类文件的查找路径;2. 在控制台程序开发中,体会到Eclipse的可视化开发优势(自动编译、错误提示)与命令行开发的底层逻辑(手动编译、明确路径关系),同时注意到包声明的重要性——包名需与目录结构一致,否则会出现‘找不到主类’的问题;3. 错误提示实验让我熟悉了Java编译器对语法错误的反馈方式,如丢失大括号、分号会提示‘缺失符号’,大小写错误会提示‘无法解析变量’,这有助于后续快速定位代码错误;4. 关于面向对象与面向过程,通过五子棋与成绩管理系统的案例,我理解了两者的核心差异——面向过程关注‘步骤’,面向对象关注‘对象’,面向对象的封装、继承、多态特性更适合复杂项目的开发与维护;5. 实验中遇到的‘找不到主类’问题,让我学会从路径、包声明、编译状态三个维度排查问题,提升了问题解决能力。未来我会进一步学习面向对象的核心特性,尝试使用uni-app开发简单APP,深化Java开发技能。”)

\end{document}