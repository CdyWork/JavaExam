\documentclass[12pt,a4paper]{article}
\usepackage{ctex}
\usepackage{geometry}
\usepackage{graphicx}
\usepackage{setspace}
\usepackage{listings}
\usepackage{listingsutf8}
\usepackage{xcolor}
\usepackage{hyperref}
\usepackage{float} 


\geometry{left=3cm,right=3cm,top=2.5cm,bottom=2.5cm}

% 代码样式
\lstset{
    language=Java,
    inputencoding=utf8,
    basicstyle=\ttfamily\small,
    keywordstyle=\color{blue}\bfseries,
    commentstyle=\color{green!50!black},
    stringstyle=\color{orange},
    showstringspaces=false,
    numbers=left,
    numberstyle=\tiny,
    breaklines=true,
    frame=single
}

\begin{document}

% ================= 封面 =================
\begin{titlepage}
\centering

% 校徽 Logo 
\makebox[\textwidth][c]{%
  \includegraphics[height=4cm]{fengmian.png}%
}
\vspace*{2cm}

% 学院名称
{\zihao{1}\heiti 信息科学与工程学院}\\[1cm]

% 学年学期
{\zihao{4} 2025---2026 \kaishu学年第一学期}\\[1.5cm]

% 报告标题
\makebox[\textwidth][c]{%
  \includegraphics[height=2cm]{shiyanbaogao.png}%
}
\\[2em] % 空行
% 实验基本信息表
\zihao{4} 
\renewcommand{\arraystretch}{1.8} % 表格行距
\begin{tabular}{rl}
\heiti 课程名称: & \underline{\makebox[18em][c]{\fangsong Java 编程技术}} \\
\vspace{1cm}
\heiti 实验名称: & \underline{\makebox[18em][c]{\fangsong 关键字索引}} \\
\kaishu 专  业  班  级 & \underline{\makebox[18em][c]{\kaishu 通信一班}} \\
\kaishu 学  生  学  号 & \underline{\makebox[18em][c]{\kaishu 202300120317}} \\
\kaishu 学  生  姓  名 & \underline{\makebox[18em][c]{\kaishu 陈都阳}} \\
\kaishu 实  验  时  间 & \underline{\makebox[18em][c]{\kaishu 2025年9月16日}} \\
\end{tabular}

\vfill
\end{titlepage}

% ================= 正文 =================
{\zihao{2}【关键字索引】}\\

\setstretch{1.2}
\noindent
\textbf{abstract}:声明抽象类或抽象方法。\\
\textbf{assert}:用于调试时的断言检查。\\
\textbf{boolean}:声明布尔类型变量(true 或 false)。\\
\textbf{break}:跳出当前循环或 switch 语句。\\
\textbf{byte}:声明 8 位有符号整数类型。\\
\textbf{case}:switch 语句中的分支标签。\\
\textbf{catch}:捕获异常。\\
\textbf{char}:声明字符类型(16 位 Unicode)。\\
\textbf{class}:声明一个类。\\
\textbf{const}:保留关键字,未使用。\\
\textbf{continue}:跳过当前循环的剩余部分并进入下一次循环。\\
\textbf{default}:switch 语句中的默认分支。\\
\textbf{do}:与 while 一起使用,构成 do-while 循环。\\
\textbf{double}:声明双精度浮点数类型。\\
\textbf{else}:if 语句的“否则”分支。\\
\textbf{enum}:定义枚举类型。\\
\textbf{extends}:表明一个类继承自另一个类。\\
\textbf{final}:声明常量、不可继承的类或不可重写的方法。\\
\textbf{finally}:定义在异常处理后一定会执行的代码块。\\
\textbf{float}:声明单精度浮点数类型。\\
\textbf{for}:定义 for 循环结构。\\
\textbf{goto}:保留关键字,未使用。\\
\textbf{if}:条件语句的起始关键字。\\
\textbf{implements}:声明一个类实现接口。\\
\textbf{import}:导入包中的类或接口。\\
\textbf{instanceof}:测试对象是否为某个类的实例。\\
\textbf{int}:声明整数类型(32 位)。\\
\textbf{interface}:定义接口。\\
\textbf{long}:声明长整型(64 位)。\\
\textbf{native}:声明一个本地方法(由其他语言实现)。\\
\textbf{new}:创建新对象实例。\\
\textbf{null}:表示空引用。\\
\textbf{package}:定义类所在的包。\\
\textbf{private}:声明私有访问权限。\\
\textbf{protected}:声明受保护访问权限。\\
\textbf{public}:声明公有访问权限。\\
\textbf{return}:从方法返回结果。\\
\textbf{short}:声明 16 位整数类型。\\
\textbf{static}:声明静态成员。\\
\textbf{strictfp}:限制浮点计算的精度和舍入。\\
\textbf{super}:引用父类的成员或构造方法。\\
\textbf{switch}:多分支选择语句。\\
\textbf{synchronized}:声明同步方法或代码块。\\
\textbf{this}:引用当前对象。\\
\textbf{throw}:抛出异常。\\
\textbf{throws}:声明可能抛出的异常类型。\\
\textbf{transient}:声明不被序列化的成员变量。\\
\textbf{try}:定义可能抛出异常的代码块。\\
\textbf{void}:声明方法无返回值。\\
\textbf{volatile}:声明变量在多线程中保持可见性。\\

\end{document}

